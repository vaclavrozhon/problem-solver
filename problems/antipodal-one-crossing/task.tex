% Improving Dvořák's 3/8 bound on color changes
% \newcommand{\cube}{Q_n}
% \newcommand{\V}{\{0,1\}^n}
% \newcommand{\E}{E}
% \newcommand{\col}{\chi}
% \newcommand{\R}{\mathsf{R}}
% \newcommand{\B}{\mathsf{B}}
% \newcommand{\switches}{\mathrm{sw}}

\section*{Model and task}

\paragraph{Setup.}
Let $\cube$ be the $n$-cube. Consider an \emph{arbitrary} 2-coloring $\col:\E(\cube)\to\{\R,\B\}$ (no antipodality constraint). For a path $P=(e_1,\dots,e_L)$, let $\switches(P)$ be the number of color changes along $P$.

\paragraph{Objective (beat $3/8$).}
Dvořák proved that there is an antipodal path with at most $\big(\tfrac{3}{8}+o(1)\big)n$ switches. Prove any constant improvement:
\[
\exists\,\delta>0\ \text{(absolute)}\quad
\forall\,n\ \ \text{every 2-coloring has antipodal } P
\text{ with }\switches(P)\le \Big(\tfrac{3}{8}-\delta\Big)n.
\]
Even $\delta=10^{-4}$ would be significant.

\section*{Context and prior results}

\paragraph{Baseline bounds.}
Leader--Long showed a monochromatic geodesic of length $\ge \lceil n/2\rceil$, implying an antipodal path with at most $n/2$ switches. Dvořák sharpened this to $(3/8+o(1))n$ by a refined random-geodesic and $Q_3$-patching argument.


\section*{Possible approaches and technical ideas}

\paragraph{Important note.}
The items below should be treated as results of brainstorming, take all of them with a grain of salt. 


\paragraph{1.\ Expander vs.\ sparse-cut dichotomy at block scale.}
Partition coordinates into blocks $J_1,\dots,J_T$ of size $m$ (e.g., $m=128$). Let $S_J=\{x:\sum_{i\in J}x_i\equiv 0\pmod 2\}$, and denote by $\beta_J$ the fraction of $J$-direction edges that are blue (across $\delta S_J$).
\begin{itemize}
  \item \textbf{Case A (many biased blocks).} For at least $(1-\theta)$ fraction of blocks, $\beta_J\notin[\tfrac12-\varepsilon,\tfrac12+\varepsilon]$. In each biased block, a random start and random ordering of coordinates yields (by Chernoff) a run that flips all $|J|$ coordinates with only $O(\varepsilon |J|)$ off-color edges. Use 4-cycle commuting to cluster the defects, so processing all blue-majority blocks in one blue phase and red-majority blocks in one red phase costs
  \[
  \text{switches} \ \le\ \underbrace{T}_{\text{phase boundaries}} \;+\; O(\varepsilon n).
  \]
  Choosing $m$ large and $\varepsilon\ll 1/m$ gives $\le (1/m+\varepsilon)n<0.375n$.
  \item \textbf{Case B (few biased blocks).} For at least $(1-\theta)$ fraction of blocks, both colors occupy a constant fraction of \emph{every} block boundary. Summing the block matchings shows that at least one color subgraph has spectral gap $\Omega(1)$ (matrix Chernoff on the sum of matchings), hence linear conductance and, in particular, a \emph{monochromatic} antipodal path (0 switches).
\end{itemize}
\emph{Deliverable sublemma:} If a color occupies at least an $\varepsilon$-fraction of each $\delta S_J$ for $(1-\theta)$ of the blocks, then that color has conductance $\Omega(\varepsilon)$ in the whole cube.

\paragraph{2.\ $Q_3$ density improvement over $1/2$.}
Dvořák’s method analyzes a random geodesic and counts ``good'' $Q_3$ subcubes that allow local color-change savings. Prove that in any 2-coloring, the fraction of good $Q_3$’s exceeds $1/2+\gamma$ for some absolute $\gamma>0$. Then the expected number of savings improves by $\Omega(\gamma n)$, pushing the constant below $3/8$.
\begin{itemize}
  \item Candidate route: classify $Q_3$ colorings; show certain ``worst'' patterns cannot tile the cube at density $1/2$ due to parity/overlap constraints on shared 4-cycles. A linear-program bound over pattern frequencies may certify $\gamma>0$.
\end{itemize}

\paragraph{3.\ Local surgery with better defect-to-switch conversion.}
Along a nearly monochromatic segment containing a fraction $\rho$ of off-color edges, one can iteratively remove $RBR$ (or $BRB$) peaks using 4-cycles on the two involved coordinates. A careful accounting shows the number of resulting switches is \emph{twice} the number of disjoint defect clusters. Sharpen this combinatorial lemma to show
\[
\text{switches} \ \le\ (2+o(1))\,\rho n,
\]
uniformly over concatenated segments. Coupled with any scheme producing $\rho\le 0.18-\epsilon$, this already beats $3/8=0.375$.

\paragraph{4.\ Fourier/junta stability for sparse colored boundaries.}
Let $S\subset\V$ minimize the blue fraction across a cut: $|E_\B(S,\overline{S})|$ small relative to $|E(S,\overline{S})|$. Prove a junta-type stability: most boundary measure concentrates on a small set of coordinates $I$, and across those directions one color dominates. This yields a long single-color burst on $I$ with $o(n)$ switches, then recurse on $I^c$. With $\eta=O(\sqrt{\varepsilon})$ stability, you obtain $(\varepsilon+\eta)n$ switches; choosing constants delivers any fixed gain below $3/8$.

\paragraph{5.\ Refined random-geodesic coupling.}
In Dvořák’s proof, choices of coordinate order are uniform. Bias the distribution toward orders that maximize the overlap with biased blocks or good $Q_3$’s (via an exposure martingale and a size-biased pick of next coordinate). Freedman-type concentration can control the variance while increasing the expected savings per step by a fixed constant, giving $\tfrac{3}{8}-\delta$.

\paragraph{6.\ Hybrid multi-scale strategy.}
Run the process in \emph{phases}: at scale $m$ apply block-parity compression; on the residual set of coordinates with weak block bias, switch to the $Q_3$ density argument; if both fail to yield a strict gain, the spectral lemma triggers the expander case (0 switches). The minimum gain across phases is a fixed $\delta>0$.
