# Problem definition and your task

Fix an integer $n\ge 1$ and let $[n]=\{1,2,\dots,n\}$. Consider the complete graph $G$ on the vertex set

$$
V(G)=[n]\times[n]\times[n].
$$

For two distinct vertices

$$
u=(x_1,y_1,z_1),\qquad v=(x_2,y_2,z_2),
$$

define the **sign pattern**

$$
\sigma(u,v)=\big(\operatorname{sgn}(x_1-x_2),\,\operatorname{sgn}(y_1-y_2),\,\operatorname{sgn}(z_1-z_2)\big)\in\{-1,0,+1\}^3\setminus\{(0,0,0)\}.
$$

The **type** of the undirected edge $e=uv$ is the equivalence class

$$
T(e)=[\sigma(u,v)]\quad\text{modulo global sign }(\sigma\sim -\sigma),
$$

so there are $(3^3-1)/2=13$ types. Let $J(T)\subseteq\{x,y,z\}$ be the coordinates where $T$ is nonzero, and $E(T)=\{x,y,z\}\setminus J(T)$ the coordinates where it is zero.

Define the color map $c_3:E(G)\to\mathcal L$ as follows. For $i\in J(T)$ (the two points $u,v$ disagree on the coordinate), record $m_i:=\min\{u_i,v_i\}$. For $i\in E(T)$, don't record anything. The list of length 3 that contains either the minimum coordinate, or "unspecified", together with the type of the point-pair, is the color of the edge. Check the $d=2$ solution to see how this works. 

For $U\subseteq V(G)$, let $G[U]$ denote the induced subgraph. An edge $e\in E(G[U])$ has a **unique color in $U$** if no other edge of $G[U]$ has the same label $c_3(e)$.

# Your task

Show that for every vertex set $U\subseteq [n]^3$ with $|U|\ge 2$, there exist $u_1,u_2\in U$ such that the edge $u_1u_2$ has a color that appears **exactly once** in $G[U]$ under the coloring $c_3$ above.


Next, you are given a file with a proof for the analogous question in $d=2$. 


\documentclass[11pt]{article}
\usepackage{amsmath,amsthm,amssymb,mathtools}
\usepackage[margin=1in]{geometry}

\newtheorem{theorem}{Theorem}
\newtheorem{lemma}[theorem]{Lemma}
\theoremstyle{definition}
\newtheorem{definition}[theorem]{Definition}
\newtheorem{fact}[theorem]{Fact}
\newtheorem{remark}[theorem]{Remark}
\newtheorem{claim}[theorem]{Claim}

\newcommand{\NE}{\mathrm{NE}}
\newcommand{\set}[1]{\left\{\,#1\,\right\}}

\begin{document}

\section*{Problem and main theorem}

Fix $n\ge 1$ and write $[n]=\{1,2,\dots,n\}$. Consider the complete graph $G$ on
$V(G)=[n]\times[n]$. For distinct vertices $v_1=(a_1,b_1)$ and $v_2=(a_2,b_2)$, the edge
$e=v_1v_2$ is of type
\begin{itemize}
  \item \textbf{1:} $(a_1-a_2)(b_1-b_2)>0$,
  \item \textbf{2:} $(a_1-a_2)(b_1-b_2)<0$,
  \item \textbf{3:} $a_1=a_2$ (vertical),
  \item \textbf{4:} $b_1=b_2$ (horizontal),
\end{itemize}
with color
\[
c(e)=
\begin{cases}
(1,\min\{a_1,a_2\},\min\{b_1,b_2\}) & \text{type 1},\\
(2,\min\{a_1,a_2\},\min\{b_1,b_2\}) & \text{type 2},\\
(3,\min\{b_1,b_2\}) & \text{type 3},\\
(4,\min\{a_1,a_2\}) & \text{type 4}.
\end{cases}
\]
For $U\subseteq V(G)$, write $G[U]$ for the induced subgraph. An edge $e\in E(G[U])$ has a
\emph{unique color in $U$} if no other edge of $G[U]$ has the same label $c(e)$.

\begin{theorem}\label{thm:unique}
For every $U\subseteq [n]\times[n]$ with $|U|\ge 2$, the graph $G[U]$ contains an edge whose
color under $c$ appears exactly once in $G[U]$.
\end{theorem}

From now on, fix $U$ and assume the following: 
\[
\textbf{(H)}\qquad \text{No color is unique in }G[U].
\]

\section*{Basic notation and counting}

For $p=(x,y)\in U$, define its strict north--east set
\[
\NE(p):=\set{q=(x',y')\in U:\ x'>x\ \text{and}\ y'>y}.
\]
We write $p\prec q$ iff $q\in\NE(p)$ (both coordinates increase).

\begin{fact}[Type~1 multiplicity]\label{fact:t1}
For $(\alpha,\beta)\in[n]^2$,
\[
\#\{e\in E(G[U]): c(e)=(1,\alpha,\beta)\}
= \mathbf 1_{(\alpha,\beta)\in U}\cdot |\NE(\alpha,\beta)|.
\]
In particular, $(1,\alpha,\beta)$ is unique in $U$ iff $(\alpha,\beta)\in U$ and $|\NE(\alpha,\beta)|=1$.
\end{fact}

\begin{fact}[Type~2 multiplicity]\label{fact:t2}
For $(\alpha,\beta)\in[n]^2$ put
\[
r(\alpha,\beta)=\#\{(\alpha,y)\in U:\ y>\beta\},\qquad
s(\alpha,\beta)=\#\{(x,\beta)\in U:\ x>\alpha\}.
\]
Then the number of edges colored $(2,\alpha,\beta)$ in $G[U]$ equals $r(\alpha,\beta)\,s(\alpha,\beta)$.
Thus $(2,\alpha,\beta)$ is unique iff $r(\alpha,\beta)=s(\alpha,\beta)=1$.
\end{fact}

\section*{Lexicographic maximizer}

\begin{definition}[Eligible and lexicographic maximizer]\label{def:lex}
A point $p\in U$ is \emph{eligible} if $\NE(p)\neq\varnothing$.
Order pairs by $(y,x)\le_{\mathrm{lex}}(y',x')$ iff $y<y'$ or $y=y'$ and $x\le x'$.
A point $a\in U$ is a \emph{lexicographic maximizer among eligible points} if
\[
\NE(a)\neq\varnothing\quad\text{and}\quad (a_y,a_x)=\max\nolimits_{\le_{\mathrm{lex}}}
 \set{(y,x): (x,y)\in U,\ \NE(x,y)\neq\varnothing}.
\]
\end{definition}

\begin{lemma}[Lexicographic maximality $\Rightarrow$ no comparable pair]\label{lem:lexheight}
If $a$ is as in Definition~\ref{def:lex}, then $\NE(q)=\varnothing$ for every $q\in \NE(a)$.
Hence $\NE(a)$ is an antichain for $\prec$ (no strictly NE–comparable pair).
\end{lemma}

\begin{proof}
If $q\in\NE(a)$ had $\NE(q)\neq\varnothing$, then $(q_y,q_x)>(a_y,a_x)$ lexicographically,
contradicting maximality. If $r,s\in \NE(a)$ with $r\prec s$, then $\NE(r)\ni s\neq\varnothing$, again a contradiction.
\end{proof}

\section*{Levels in $\NE(a)$ and \emph{row/column–good}}

For $S\subseteq\mathbb{Z}^2$ write $S[y]=\{(x,y)\in S\}$ and $S[x]=\{(x,y)\in S\}$.
For a point $a$ with $\NE(a)\neq\varnothing$ define (when they exist)
\[
Y_{\min}(a):=\min\{\,y>a_y:\ \NE(a)[y]\neq\varnothing\,\},\qquad
X_{\min}(a):=\min\{\,x>a_x:\ \NE(a)[x]\neq\varnothing\,\},
\]
(the \emph{bottommost row} and \emph{leftmost column} of $\NE(a)$).

\begin{definition}[Row–good / Column–good]\label{def:row-good}
Let $a$ be lex–maximal among eligible points.

\smallskip\noindent
\emph{Row–good:}
There exists $\beta=Y_{\min}(a)$ such that $\#\,\NE(a)[\beta]\ge 2$, and if we list the $x$–coordinates of
$\NE(a)[\beta]$ increasingly as $x_1<\cdots<x_t$ ($t\ge 2$) and set
\[
b_1=(x_{t-1},\beta)\quad\text{(second–rightmost)},\qquad
b_2=(x_t,\beta)\quad\text{(rightmost)},
\]
then
\[
\text{(RG)}\qquad \forall q\in \NE(a)\setminus\{b_1,b_2\}:\quad x(q)<x(b_1)\ \ \text{and}\ \ y(q)\ge \beta.
\]
Equivalently: $b_1,b_2$ are the \emph{last two} points of $\NE(a)$—every other point lies strictly to the left of $b_1$ and on row $\beta$ or above.

\smallskip\noindent
\emph{Column–good:} the vertical analogue with $\alpha=X_{\min}(a)$, $d_1,d_2$ the \emph{second–topmost} and \emph{topmost} points in $\NE(a)[\alpha]$, and
\[
\text{(CG)}\qquad \forall q\in \NE(a)\setminus\{d_1,d_2\}:\quad y(q)<y(d_1)\ \ \text{and}\ \ x(q)\ge \alpha.
\]
\end{definition}

\section*{Dichotomy}

\begin{definition}[Corner in $\NE(a)$]\label{def:corner}
Let $a$ be lex–maximal among eligible points. We say $\NE(a)$ contains a \emph{corner} if there exist
\[
(\alpha,\beta),\quad (\alpha',\beta),\quad (\alpha',\beta')\in \NE(a)
\]
with
\[
a_x<\alpha<\alpha',\qquad a_y<\beta'<\beta.
\]
\end{definition}

\medskip

\noindent\textbf{Additional notation.}
For $q=(x(q),y(q))$ we continue to write $x(q),y(q)$ for its coordinates.

\subsection*{Partially good pairs (formalizing the sketch)}

\begin{definition}[Partially row–good / column–good]\label{def:partial}
Let $a$ be lex–maximal among eligible points.

\smallskip\noindent
Two points $b_1,b_2\in\NE(a)$ on the same row $y=\beta$ with $x(b_1)<x(b_2)$ are \emph{partially row–good}
if
\begin{align*}
\text{(PR1)}\ &\text{$b_1,b_2$ are the last two points of $\NE(a)$ on row $\beta$;}\\
\text{(PR2)}\ &\forall q\in\NE(a)\setminus\{b_1,b_2\}:\ y(q)\ge\beta \ \Rightarrow\ x(q)<x(b_1).
\end{align*}
Equivalently, this is exactly (RG) with the sole relaxation that points with $y<\beta$ may exist anywhere.

\smallskip\noindent
Two points $d_1,d_2\in\NE(a)$ on the same column $x=\alpha$ with $y(d_1)<y(d_2)$ are \emph{partially column–good}
if
\begin{align*}
\text{(PC1)}\ &\text{$d_1,d_2$ are the topmost two points of $\NE(a)$ on column $\alpha$;}\\
\text{(PC2)}\ &\forall q\in\NE(a)\setminus\{d_1,d_2\}:\ x(q)\ge\alpha \ \Rightarrow\ y(q)<y(d_1).
\end{align*}
Equivalently, this is (CG) with the sole relaxation that points with $x<\alpha$ may exist anywhere.
\end{definition}

\begin{remark}[On the intended meaning in the sketch]\label{rem:meaning}
The phrasing ``no points are above $b_1$ and above $b_2$'' in the sketch is formalized by (PR2):
above the row $\beta$ there are no points of $\NE(a)$ at columns $\ge x(b_1)$. This was the property actually used
to force $r(\alpha,\beta)=1$ in the type–2 counting (Fact~\ref{fact:t2}).
\end{remark}

\subsection*{Propagation sublemma (rigorous version of the sketch)}

\begin{lemma}[Propagation for partially row–good pairs]\label{lem:prop-row}
Assume \emph{(H)} and let $a$ be lex–maximal among eligible points.
Let $b_1,b_2\in\NE(a)$ be partially row–good on row $\beta$ with $x(b_1)<x(b_2)$.
Then either
\begin{enumerate}
\item[\emph{(i)}] $\NE(a)[y]=\varnothing$ for every $y<\beta$ (hence $a$ is row–good), or
\item[\emph{(ii)}] there exist $b_1',b_2'\in\NE(a)$ on some row $\beta'<\beta$ that are partially row–good.
\end{enumerate}
\end{lemma}

\begin{proof}
If $\NE(a)[y]=\varnothing$ for all $y<\beta$ we are in case (i) and done.
Otherwise, let
\[
\beta':=\max\{\,y<\beta:\ \NE(a)[y]\neq\varnothing\,\}
\]
be the highest row strictly below $\beta$, and within that row choose
\[
c_1\in \NE(a)[\beta']\quad\text{with}\quad x(c_1)=\min\{\,x:\ (x,\beta')\in \NE(a)\,\}.
\]
By Lemma~\ref{lem:lexheight}, every point of $\NE(a)$ has empty NE–set. In particular,
$c_1$ cannot lie strictly to the left of $b_2$ since then $b_2\in\NE(c_1)$; hence
\begin{equation}\label{eq:c1rightofb2}
x(c_1)\ge x(b_2).
\end{equation}

We split according to whether $c_1$ is directly below $b_2$ (same column) or strictly to its right.

\smallskip
\noindent\emph{Case A: $x(c_1)=x(b_2)$.}
Consider the type–2 edge $b_1c_1$. Its color is $(2,\alpha,\gamma)$ with
$\alpha=x(b_1)$ and $\gamma=\beta'$. By the choice of $\beta'$ there is no point of $U$ in
column $x(b_1)$ with $y\in(\beta',\beta)$ (such a point would belong to $\NE(a)$ since $x(b_1)>a_x$).
Moreover, by (PR2) there is no $q\in \NE(a)\setminus\{b_1,b_2\}$ with $y(q)\ge\beta$ and $x(q)\ge x(b_1)$;
hence in column $x(b_1)$ there is no point of $U$ with $y>\beta$ (the only point at height $\ge \beta$ is $b_1$ itself at $y=\beta$).
Consequently
\[
r(\alpha,\gamma)=\#\{(x(b_1),y)\in U:\ y>\beta'\}=1
\]
(the unique witness is $b_1$ at height $\beta$). Since \emph{(H)} forbids uniqueness of $(2,\alpha,\gamma)$,
Fact~\ref{fact:t2} implies $s(\alpha,\gamma)\ge 2$. One of these $s$ witnesses is $c_1$ itself, and because $c_1$
was chosen leftmost on $\beta'$, the second one must have strictly larger $x$; call it $c_2$ (so $x(c_2)>x(c_1)$).

Now consider $b_2c_2$. Its color is $(2,\alpha',\gamma)$ with $\alpha'=x(b_2)$ and the same $\gamma=\beta'$.
By the definition of $\beta'$ there is no point of $U$ in column $x(b_2)$ with $y\in(\beta',\beta)$, and since $x(b_2)\ge x(b_1)$,
(PR2) rules out any $q\in \NE(a)\setminus\{b_1,b_2\}$ with $y(q)\ge\beta$ and $x(q)\ge x(b_2)$; thus there is no point of $U$
in that column with $y>\beta$. Therefore
$r(\alpha',\gamma)=1$ (the only point above height $\beta'$ in that column is $b_2$), so $s(\alpha',\gamma)\ge 2$.
Because $c_2$ is one $\beta'$–witness with $x>\alpha'$, there exists $c_3\in \NE(a)[\beta']$ with $x(c_3)>x(c_2)$.

Let $b_2'$ be the rightmost point of $\NE(a)[\beta']$ and $b_1'$ the second–rightmost point of $\NE(a)[\beta']$.
By construction $(\text{PR1})$ holds for $b_1',b_2'$.
For $(\text{PR2})$: there are no points of $\NE(a)$ on rows $y\in(\beta',\beta)$ by the choice of $\beta'$;
and for rows $y\ge\beta$ there are no points with $x\ge x(b_1)\le x(b_1')$ by (PR2) for the original
pair $b_1,b_2$. Hence every $q\in\NE(a)\setminus\{b_1',b_2'\}$ with $y(q)\ge\beta'$ satisfies $x(q)<x(b_1')$.
Thus $b_1',b_2'$ are partially row–good, verifying case (ii).

\smallskip
\noindent\emph{Case B: $x(c_1)>x(b_2)$.}
Consider $b_2c_1$, which has color $(2,\alpha',\gamma)$ with $\alpha'=x(b_2)$ and $\gamma=\beta'$.
Exactly the same reasoning as in Case~A gives $r(\alpha',\gamma)=1$, hence $s(\alpha',\gamma)\ge 2$.
One witness is $c_1$; since $c_1$ is leftmost on $\beta'$, the second witness $c_2$ must satisfy $x(c_2)>x(c_1)$.
Define $b_2'$ as the rightmost point and $b_1'$ as the second–rightmost point of $\NE(a)[\beta']$.
Then (PR1) holds by definition, and (PR2) follows verbatim as in Case~A. Hence $b_1',b_2'$ are partially row–good.
\end{proof}

\begin{remark}[Minor tidy–up matching the sketch]\label{rem:tidy-up}
In the sketch, after producing $c_2,c_3$ it is stated that one can set $b_1'b_2'=c_2,c_3$.
To ensure (PR1) literally (``last two''), one should in general relabel
$b_1',b_2'$ as the \emph{second–rightmost} and \emph{rightmost} points of $\NE(a)[\beta']$.
This harmless relabeling is the only additional step we make for full rigor.
\end{remark}

\begin{lemma}[Propagation for partially column–good pairs]\label{lem:prop-col}
Assume \emph{(H)} and $a$ lex–maximal among eligible points.
Let $d_1,d_2\in\NE(a)$ be partially column–good on column $\alpha$ with $y(d_1)<y(d_2)$.
Then either
\begin{enumerate}
\item[\emph{(i)}] $\NE(a)[x]=\varnothing$ for every $x<\alpha$ (hence $a$ is column–good), or
\item[\emph{(ii)}] there exist $d_1',d_2'\in\NE(a)$ on some column $\alpha'<\alpha$ that are partially column–good.
\end{enumerate}
\end{lemma}

\begin{proof}
By symmetry of $x\leftrightarrow y$ and $(\text{PR1}),(\text{PR2})\leftrightarrow(\text{PC1}),(\text{PC2})$,
the proof is identical to Lemma~\ref{lem:prop-row}, with Fact~\ref{fact:t2} used in the transposed roles.
\end{proof}

\subsection*{Existence of a starting partially good pair and the dichotomy}

\begin{lemma}[Dichotomy: each lex–maximal $a$ is row–good or column–good]\label{lem:dichotomy-strong}
Assume \emph{(H)} and let $a$ be lexicographically maximal among eligible points.
Then $a$ is row–good or column–good.
\end{lemma}

\begin{proof}
We first note that $\#\NE(a)\ge 2$ under (H): if $\#\NE(a)=1$, then by Fact~\ref{fact:t1}
the color $(1,a_x,a_y)$ appears exactly once (on the unique type–1 edge from $a$), contradicting (H).

Let $p\in\NE(a)$ be a point of maximal $y$ (i.e.\ on the topmost row of $\NE(a)$).
If $\NE(a)[y(p)]$ contains at least two points, let $b_2$ be the rightmost
and $b_1$ the second–rightmost point of that row. Then $b_1,b_2$ are partially row–good:
(PR1) holds by choice; for (PR2), there is no point of $\NE(a)$ with $y>\,y(p)$ and
every other point on $y(p)$ lies to the left of $b_1$.
Applying Lemma~\ref{lem:prop-row} iteratively, either we immediately conclude that no lower row exists
and $a$ is row–good, or we propagate downward until we arrive at the bottommost row,
at which point the pair is row–good by definition. This iteration terminates because the row index
decreases strictly at each step and there are finitely many rows. This settles the case $\#\NE(a)[y(p)]\ge 2$.

It remains to treat the case where $p$ is the unique point on its row.
Let $\beta'$ be the next highest level below $y(p)$ with $\NE(a)[\beta']\neq\varnothing$.
We claim that every point on row $\beta'$ lies on or to the right of $p$ (i.e.\ has $x\ge x(p)$):
indeed, if $q\in\NE(a)[\beta']$ had $x(q)<x(p)$, then $p\in\NE(q)$, which contradicts
Lemma~\ref{lem:lexheight}. There are two subcases.

\smallskip
\noindent\emph{Subcase 1: there is $q_1\in\NE(a)[\beta']$ with $x(q_1)=x(p)$ (``directly below'' $p$).}
If $\NE(a)[\beta']=\{q_1\}$, then the pair $d_1=q_1$, $d_2=p$ is partially column–good:
(PC1) holds because these are the topmost two points in column $x(p)$;
(PC2) holds because there is no row strictly between $\beta'$ and $y(p)$,
there is no point on row $\beta'$ with $x\ge x(p)$ other than $q_1$ itself,
and $p$ is alone on row $y(p)$. Applying Lemma~\ref{lem:prop-col} iteratively,
either we immediately conclude that no more left columns exist and $a$ is column–good,
or we propagate left until reaching the leftmost column, at which point the pair is column–good.
This iteration terminates because the column index decreases strictly at each step and there are finitely many columns.

If instead $\NE(a)[\beta']$ contains another point $q_2$ to the right of $q_1$,
then (H) plus Fact~\ref{fact:t2} applied to the type–2 edge $pq_2$ forces the existence of
$q_3\in\NE(a)[\beta']$ with $x(q_3)>x(q_2)$ (formally, $r(x(p),\beta')=1$ and so $s(x(p),\beta')\ge 2$;
choosing $q_2$ leftmost to the right of $x(p)$ guarantees $x(q_3)>x(q_2)$).
Relabel the rightmost two points on row $\beta'$ as $b_1,b_2$; then $b_1,b_2$ are partially row–good
and we conclude by Lemma~\ref{lem:prop-row} as in the first paragraph (again the iteration terminates as above).

\smallskip
\noindent\emph{Subcase 2: there is no point directly below $p$ on $\beta'$.}
Pick any $q_1\in\NE(a)[\beta']$; we know $x(q_1)>x(p)$. Then the type–2 edge $pq_1$ has color $(2,x(p),\beta')$
with $r(x(p),\beta')=1$ (the only point above height $\beta'$ in column $x(p)$ is $p$),
so (H) and Fact~\ref{fact:t2} yield another $q_2\in\NE(a)[\beta']$ with $x(q_2)>x(q_1)$
(again choose $q_1$ leftmost on $\beta'$ to force $x(q_2)>x(q_1)$).
Relabel the rightmost two points on $\beta'$ as $b_1,b_2$, which are partially row–good,
and finish via Lemma~\ref{lem:prop-row}$\,$— the iteration terminates for the same reason as above.

\smallskip
In all cases, we obtain that $a$ is row–good or column–good.
\end{proof}

\begin{remark}[Explicitly pointing out the sketch’s gaps]\label{rem:gaps}
\leavevmode
\begin{itemize}
\item In the sublemma part of the sketch, after producing $c_2,c_3$ it is said to set $b_1'b_2'=c_2,c_3$.
To certify the ``last two'' requirement, one must relabel $b_1',b_2'$ as the \emph{rightmost two} points
on the row $\beta'$ (Remark~\ref{rem:tidy-up}). This does not change the argument’s structure.
\item In the ``starting pair'' part, the phrase ``there is $q_1$ to the right of $p$'' should be
understood as: $q_1$ lies on the \emph{next highest} row $\beta'$ below $y(p)$ and satisfies $x(q_1)>x(p)$.
With this reading, the verification of the partial row–good property is immediate from the choice of $\beta'$.
\end{itemize}
\end{remark}

\bigskip

\section*{Final proof of Theorem~\ref{thm:unique} (as provided, with small auxiliary lemmas)}

\paragraph{Convention.}
In this section, when we say ``lex–maximal point'', we mean \emph{lex–maximal among eligible points} with respect to the order $(y,x)$.

\begin{lemma}[Lex–max eligible exists inside a cone]\label{lem:cone-max-exists}
Let $p\in U$, and let $\overline{\NE}(p)=\{\,q\in U:\ x(q)\ge x(p),\ y(q)\ge y(p)\,\}$.
If $\overline{\NE}(p)$ contains an eligible point, then it contains a lex–maximal eligible point (with respect to $(y,x)$).
\end{lemma}

\begin{proof}
$\overline{\NE}(p)$ is finite; the set of eligible points inside it is finite and nonempty. A lexicographic maximum exists.
\end{proof}

\begin{lemma}[Local antichain from cone–maximality]\label{lem:local-anti}
Fix $p\in U$ and let $c$ be lex–maximal among eligible points in $\overline{\NE}(p)$.
Then $\NE(q)=\varnothing$ for every $q\in \NE(c)$ (equivalently, $\NE(c)$ is an antichain).
\end{lemma}

\begin{proof}
If $q\in \NE(c)$ had $\NE(q)\neq\varnothing$, then $q$ would be eligible and $(y(q),x(q))>(y(c),x(c))$.
Also $q\in\overline{\NE}(p)$ since $q\ge c\ge p$ coordinatewise. This contradicts the lex–maximality of $c$ among eligible points in $\overline{\NE}(p)$.
\end{proof}

\begin{corollary}[Local dichotomy]\label{cor:local-dichotomy}
Assume \emph{(H)}. If $c\in U$ satisfies $\NE(c)\neq\varnothing$ and $\NE(q)=\varnothing$ for all $q\in\NE(c)$, then $c$ is row–good or column–good (with $a$ replaced by $c$ in Definition~\ref{def:row-good}).
\end{corollary}

\begin{proof}
The proof of Lemma~\ref{lem:dichotomy-strong} uses only (i) $\NE(c)\neq\varnothing$, (ii) $\NE(c)$ is an antichain, and (iii) \emph{(H)}. Under these assumptions the same argument applies verbatim.
\end{proof}

\begin{lemma}[No eligible point $\Rightarrow$ a unique color]\label{lem:no-eligible}
If $\NE(p)=\varnothing$ for every $p\in U$, then $G[U]$ contains an edge whose color is unique.
\end{lemma}

\begin{proof}
If some row contains at least two points, let $\beta$ be the minimum $y$ with $\#U[\beta]\ge 2$, and let $b_1,b_2\in U[\beta]$ be the second–rightmost and rightmost points. We claim $(4,x(b_1))$ is unique. Indeed, if there were another horizontal edge with color $(4,x(b_1))$ on some row $\beta'>\beta$, there would be points $(x(b_1),\beta')$ and $(x',\beta')$ with $x'>x(b_1)$, which makes $(x',\beta')\in \NE\big(x(b_1),\beta\big)$, contradicting that all $\NE(\cdot)$ are empty.

Otherwise, every row contains at most one point. If some column contains at least two points, let $\alpha$ be the minimum $x$ with $\#U[\alpha]\ge 2$, and let $d_1,d_2\in U[\alpha]$ be the second–topmost and topmost points ($y(d_1)<y(d_2)$). We claim $(3,y(d_1))$ is unique. If another vertical edge of color $(3,y(d_1))$ existed on some column $\alpha'\ne \alpha$, then necessarily $\alpha'>\alpha$ (by minimality of $\alpha$) and there would be points $(\alpha',y(d_1))$ and $(\alpha',y')$ with $y'>y(d_1)$. Then $(\alpha',y')\in \NE\big(\alpha,y(d_1)\big)$, again a contradiction.

Finally, if every row and every column contains at most one point, then pick any two distinct points $p=(x_1,y_1)$ and $q=(x_2,y_2)$. Since there are no eligible points, after relabeling we must have $x_1>x_2$ and $y_1<y_2$. The type–2 edge $pq$ has color $(2,\alpha,\beta)=(2,x_2,y_1)$. Because rows and columns are unique, $r(\alpha,\beta)=1$ (the unique point above $\beta$ in column $\alpha$ is $q$) and $s(\alpha,\beta)=1$ (the unique point to the right of $\alpha$ on row $\beta$ is $p$), so the color is unique by Fact~\ref{fact:t2}.
\end{proof}

\begin{proof}[Proof of Theorem~\ref{thm:unique}]
Assume \emph{(H)} for a contradiction. If $U$ contains no eligible point, Lemma~\ref{lem:no-eligible} yields a unique color, contradicting \emph{(H)}. Hence there exists an eligible point.

Pick a lexicographic maximizer $a$ among eligible points (Definition~\ref{def:lex}). By Lemma~\ref{lem:dichotomy-strong}, $a$ is row–good or column–good. By symmetry we may assume $a$ is row–good, with witnesses $(b_1,b_2)$ on the row $\beta=Y_{\min}(a)$, $x(b_1)<x(b_2)$, satisfying (RG).

\smallskip
\noindent\emph{Claim (auxiliary step).} There exists $a'\in U$, $a'\ne a$, that is row–good with $x(a')>x(a)$ and $y(a')\le y(a)$.

\smallskip
\noindent\emph{Proof of the claim.}
Because the horizontal color $c(b_1b_2)=(4,x(b_1))$ is not unique by \emph{(H)}, there exist points $c_1=(x(b_1),\gamma)$ and $c_2=(x',\gamma)$ with $x'>x(b_1)$ on some row $\gamma$; among all such rows choose $\gamma$ maximal and fix $c_1$ on that row. Since $x(c_1)=x(b_1)>a_x$ and $a$ is row–good, we have $c_1\notin \NE(a)$; thus $y(c_1)\le y(a)$. Also $b_2\in \NE(c_1)$ (because $x(b_2)>x(c_1)$ and $y(b_2)=\beta>y(c_1)$).

By Lemma~\ref{lem:cone-max-exists}, inside the weak cone $\overline{\NE}(c_1)=\{x\ge x(c_1),\,y\ge y(c_1)\}$ there exists a lex–maximal \emph{eligible} point~$c$. Note that $c\ne b_2$ because $\NE(b_2)=\varnothing$ (Lemma~\ref{lem:lexheight}). We also have $x(c)\ge x(c_1)=x(b_1)>x(a)$ and $y(c)\ge y(c_1)\le y(a)$, so $c$ lies strictly to the right of $a$ and not above it.

By Lemma~\ref{lem:local-anti}, $\NE(q)=\varnothing$ for every $q\in \NE(c)$, hence by Corollary~\ref{cor:local-dichotomy} the point $c$ is row–good or column–good. If $c$ is row–good, set $a':=c$ and the claim is proved. Otherwise, $c$ is column–good. Let $\alpha=X_{\min}(c)$ and let $d_1,d_2\in \NE(c)[\alpha]$ with $y(d_1)<y(d_2)$. (It may happen that $d_2=b_2$; in any case $d_1\notin \NE(a)$ because $a$ is row–good.)

Consider the type–2 edge $b_1d_1$, whose color is $(2,x(b_1),y(d_1))$. As in the argument in the main text (see the proof of Lemma~\ref{lem:prop-row}, Case~A), using the choice of $\gamma$, (RG), and Fact~\ref{fact:t2} we obtain $r\big(x(b_1),y(d_1)\big)=1$, hence $s\big(x(b_1),y(d_1)\big)\ge 2$. One witness is $d_1$; choosing it leftmost on its row forces another witness strictly to its right. Pushing this once more with $b_2$ in place of $b_1$ yields a third point on that row strictly further right, which contradicts the column–goodness (CG) of $c$ after relabeling the two rightmost points on that row. Therefore $c$ cannot be column–good, and $c$ must be row–good. Setting $a':=c$ proves the claim.

\smallskip
Starting from $a$, repeatedly apply the claim to obtain a strictly increasing sequence of $x$–coordinates for row–good points that are never above the previous one. Since $x\in[n]$, this process cannot continue indefinitely; thus we obtain a contradiction to \emph{(H)}. This completes the proof.
\end{proof}

\begin{remark}[Notes on earlier concerns]
(1) Applying dichotomy to $c$ is justified by Lemmas~\ref{lem:local-anti} and Corollary~\ref{cor:local-dichotomy}: we only need antichain structure within $\NE(c)$, which holds for a cone–lex–max eligible $c$. 
(2) The possibility $c=b_2$ is excluded since $b_2$ is not eligible. 
(3) The case with no eligible points is handled by Lemma~\ref{lem:no-eligible}. 
(4) The termination of the iteration is now stated explicitly.
\end{remark}

\end{document}




