%\newcommand{\OPT}{\mathrm{OPT}}
%\newcommand{\ALG}{\mathrm{ALG}}
\newcommand{\sizes}{\{1,2\}}
\newcommand{\Cache}{C}
\newcommand{\Fault}{\textsf{Fault}}
\newcommand{\Bit}{\textsf{Bit}}
\newcommand{\General}{\textsf{General}}
\newcommand{\Forced}{\textsf{Forced}}
\newcommand{\Optional}{\textsf{Optional}}
\newcommand{\UFP}{\textsc{UFP}}
\newcommand{\GC}{\textsc{Caching}}
\newcommand{\Yes}{\textsf{Yes}}
\newcommand{\No}{\textsf{No}}

\section*{Model and Variants}

\paragraph{Offline caching.}
We are given a finite set of pages $P$, an integer cache capacity $\Cache\in\mathbb{N}$, and a request sequence $r_1,r_2,\dots,r_T\in P$.
Each page $p\in P$ has an integer \emph{size} $s(p)\in\sizes$ and a \emph{fault cost} $c(p)$ determined by one of the models below.
A \emph{service} algorithm processes the sequence; if $r_t$ is in cache, no cost is incurred; otherwise (a \emph{fault}) cost $c(r_t)$ is incurred and the algorithm may evict some pages and load $r_t$, subject to the capacity constraint $\sum_{p\in S} s(p)\le \Cache$ for the current cache set $S$.

\paragraph{Policies.}
\emph{\Forced:} upon a fault, $r_t$ must be loaded into the cache (after any evictions).
\emph{\Optional:} upon a fault, the algorithm may pay $c(r_t)$ without loading $r_t$.

\paragraph{Cost models.}
\Fault: $c(p)\equiv 1$. \hspace{1em}
\Bit: $c(p)=s(p)$.\hspace{1em}
\General: arbitrary positive integral $c(p)$.

\medskip
We consider the \emph{decision} version: given a budget $L\in\mathbb{N}$, is there a valid service whose total cost is at most $L$?

\section*{Problem Statement}

Fix the page-size set $s(p)\in\sizes$ for all $p$. For any of the following four canonical variants, determine whether the decision problem is in $\mathbf{P}$ or is $\mathbf{NP}$-complete:
\begin{enumerate}
    \item \GC$_{\sizes}$--\Fault--\Forced
    \item \GC$_{\sizes}$--\Fault--\Optional
    \item \GC$_{\sizes}$--\Bit--\Forced
    \item \GC$_{\sizes}$--\Bit--\Optional
\end{enumerate}
A complete resolution of \emph{any one} of (a)--(d)---either a polynomial-time algorithm or an NP-completeness proof---meets the goal of this problem. Stronger results (e.g., \emph{strong} NP-completeness, or matching classifications across multiple variants) are welcome but not necessary.

\paragraph{Formal input/output.}
An instance is $(P,\{s(p)\}_{p\in P},\Cache, (r_t)_{t=1}^T, L)$ with $s(p)\in\sizes$.
Under the chosen policy and cost model, answer \Yes{} iff there exists a valid service whose total fault cost is $\le L$; otherwise answer \No.


\section*{Context and Prior Results (informal)}

\begin{itemize}
    \item With \emph{unit} pages, i.e., $s(p)\equiv 1$, the offline problem is solvable in polynomial time (Belady's rule).
    \item With page sizes restricted to $\{1,2,3\}$, the decision versions are known to be \emph{strongly} NP-complete in the \Fault{} and \Bit{} models, under both \Forced{} and \Optional{} policies.
    \item The gap at $\sizes=\{1,2\}$ is the key missing piece: the above hardness proofs leverage a third size class to enforce ``switching'' constraints, while the unit-size case is easy. Whether two sizes suffice for hardness, or instead admit a polynomial algorithm, is currently unresolved in these canonical settings.
\end{itemize}

\section*{Summary of Thoughts and Possible Approaches}

\paragraph{Important note}
The following text brainstorms some idea that might be useful. However, take it with a grain of salt. 

\paragraph{Why \boldmath$\{1,2\}$ is a knife-edge.}
Known reductions for $\{1,2,3\}$ exploit a size-$3$ item to force a cache profile “bottleneck” that prevents simultaneous accommodation of two size-$2$ intervals while saturating the capacity with many size-$2$ items. With only sizes $1$ and $2$, that third knob disappears. Either:
\begin{itemize}
    \item hardness still holds and can be simulated by \emph{time expansion} or \emph{block replication} (manufacturing effective ``size-$3$'' tension using multiple synchronized size-$1$ intervals), or
    \item the structure collapses to a tractable regime (e.g., via a dynamic program over small local profiles or a matroidal/laminar decomposition).
\end{itemize}

\paragraph{Algorithmic directions that look plausible.}
\begin{enumerate}
    \item \textbf{Profile DP on the path.}  
    With demands in $\{1,2\}$, the \UFP{} capacity usage at any time is an integer in $\{0,1,\dots,\Cache\}$, and every task contributes either $1$ or $2$. Over a path, one can track a \emph{config} describing how many size-$2$ tasks are “open” plus the residual $1$-demand usage. Try a DP that scans endpoints in order and updates counts with carefully bounded state. The challenge is that the set of open intervals is not determined by counts alone; you may need to encode a short \emph{fingerprint} of how open size-$2$ intervals nest across a bounded \emph{window} (e.g., via standard interval DP compressions). Showing that a small fingerprint suffices (or proving that it does not) would be decisive.
    \item \textbf{Reduction to $b$-matching on interval graphs.}  
    Interval selections under a single capacity can be modeled as a capacitated packing on an interval graph. For demands in $\{1,2\}$, explore a transformation to a weighted $b$-matching (or to two coupled matchings) by splitting each size-$2$ task into two “lanes” with coupling constraints. If the coupling can be enforced by a totally unimodular system (e.g., via interval-order matrices), integrality would yield a polynomial algorithm.
    \item \textbf{Parameterized algorithms.}  
    Design FPT algorithms in parameters like: number of \emph{capacity changes} in the optimal profile, number of \emph{overlap classes} (max depth of interval nesting), or the number of \emph{switch points} where the count of open size-$2$ tasks changes. A kernel or FPT result might then be upgraded to polynomial time by bounding the parameter in general instances---or it may strongly suggest hardness.
\end{enumerate}

\paragraph{Hardness directions that look plausible.}
\begin{enumerate}
    \item \textbf{Simulating size-$3$ without size-$3$.}  
    Start from a $\{1,2,3\}$ gadget (for \Fault{} or \Bit{}) and replace each size-$3$ interval by a synchronized \emph{bundle} of one size-$2$ and one size-$1$ interval plus an auxiliary \emph{guard construction} that forces the two to be “co-present” whenever the original size-$3$ was meant to be active. The guards must forbid the DP-like “staggering” that would otherwise circumvent the bottleneck.
    \item \textbf{Planar independent set via two sizes.}  
    Attempt a reduction from planar $\textsc{IS}$ or from \textsc{3SAT} using \emph{time-windows} that encode variable assignments and clauses with only $\{1,2\}$ demands, relying on capacity-$\Cache$ to enforce consistency. If every clause gadget can be realized using capacity-$\Cache\in\{3,4\}$ and only $\{1,2\}$ demands, NP-hardness would follow.
    \item \textbf{Weak NP-hardness via \textsc{Partition}.}  
    Encode numbers as lengths (time) while keeping demands in $\{1,2\}$; force a choice between placing a set of size-$2$ intervals versus two size-$1$ intervals whose total “span” matches a partition instance. This would not rule out pseudo-polynomial algorithms, but it would settle (a)--(d) as NP-complete.
\end{enumerate}

\paragraph{Why a positive algorithm might exist.}
Only two demand values often lead to polyhedral integrality on interval graphs. If one can show that the natural LP relaxation for \UFP{} with demands $\{1,2\}$ and path capacities is integral (perhaps after adding a \emph{bounded} family of simple inequalities), it would imply a polynomial algorithm and settle the decision problems in $\mathbf{P}$ (at least for \Fault{} and \Bit{}, where profits are uniform or proportional to demand).

\paragraph{Why hardness might still be true.}
Even with two sizes, the global coupling across time (keep-or-evict decisions must be consistent over an interval) can encode non-local constraints. If one can force \emph{switch gadgets} using only $\{1,2\}$, the classical bottleneck arguments from the $\{1,2,3\}$ constructions could carry through.
