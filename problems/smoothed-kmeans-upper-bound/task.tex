\newcommand{\X}{X}
\newcommand{\R}{\mathbb{R}}
\newcommand{\cost}{\phi}
\newcommand{\OPT}{\mathrm{OPT}}
\newcommand{\ALG}{\mathrm{ALG}}
%\newcommand{\E}{\mathbb{E}}

\section*{Algorithm, model, and task}

\paragraph{Objective.}
Fix a dataset $\X=\{x_1,\dots,x_n\}\subseteq\R^d$ and the standard $k$-means cost
\begin{equation*}
\cost(C) \;=\; \sum_{x\in \X} \min_{c\in C} \|x-c\|_2^2
\qquad\text{for a center set } C\subseteq\R^d,\ |C|=k.
\end{equation*}
Let $\OPT(k)=\min_{|C|=k}\cost(C)$ and let $\ALG(k)$ denote the cost of the clustering returned by the $k$-means++ seeding.

\paragraph{$k$-means++ seeding (Arthur--Vassilvitskii).}
Pick the first center uniformly from $\X$. Then, iteratively, sample the next center $c$ from $\X$ with probability proportional to $D(x)^2$, where $D(x)$ is the distance from $x$ to its closest previously chosen center. Stop after $k$ centers are chosen; assign each $x$ to its nearest chosen center.

\paragraph{Smoothed random-$k$ experiment.}
Given an integer $k_0\ge 1$, draw $k$ uniformly at random from $\{k_0,k_0{+}1,\dots,2k_0{-}1\}$, run $k$-means++ with this $k$ on $\X$, and set the random variable
\begin{equation*}
\alpha \;=\; \frac{\ALG(k)}{\OPT(k)} \;\ge\; 1.
\end{equation*}

\section*{The task}
Prove the strongest possible asymptotic upper bound on
\begin{equation*}
\E_{k\sim U([k_0,2k_0)),\,\text{$k$-means++}}\!\!\Big[\; \tfrac{\ALG(k)}{\OPT(k)} \;\Big],
\end{equation*}
as a function of $k_0$. Alternatively, prove the strongest possible upper bound on the value that $\tfrac{\ALG(k)}{\OPT(k)}$ achieves with probability at least $0.01$. 

\section*{Context and prior results}

\paragraph{Baseline (fixed $k$).}
For standard $k$-means++ with $k$ centers, $\E[\ALG(k)]\le O(\log k)\cdot \OPT(k)$ and there are instances with $\E[\ALG(k)]\ge \Omega(\log k)\cdot \OPT(k)$.

\paragraph{Oversampling phenomenon.}
Let $k'=(1{+}\varepsilon)k$. Results in the spirit of Theorem~5.1 of an attached paper show that sampling $k'$ centers by $D^2$-sampling (and evaluating against $\OPT(k)$) yields an expected $O(\log (1/\varepsilon))$-approximation. Intuitively, a mild amount of \emph{oversampling} dramatically sharpens the guarantee.

\paragraph{Why randomizing $k$ might help.}
As $k$ increases, $\OPT(k)$ decreases. Group the values of $OPT(k)$ by $\lfloor \log_2 (OPT(k)) \rfloor$. If there is a long "plateau" where $OPT$ stays roughly the same, up to $2$-factor, $k$-means++ effectively oversamples relative to the number of ``active'' clusters present on that scale. This could lead to a better guarantee. 


\section*{Possible approaches and technical ideas}

\paragraph{Important note}
Take the following section only as an inspiration of what might work. Take it with a huge grain of salt. 

\paragraph{1.\ Plateau decomposition of $\OPT(k)$}
Let $k_0=k^{(0)}<k^{(1)}<\cdots<k^{(m)}\le 2k_0$ be the indices at which $\OPT(\cdot)$ drops by a constant factor; define plateaus $I_j=[k^{(j)},k^{(j+1)}\!)$ on which $\OPT(k)$ is constant.
\begin{itemize}
  \item \textbf{Long vs.\ short plateaus.} Call $I_j$ \emph{long} if $|I_j|\ge k_0/\mathrm{poly}\log k_0$, else \emph{short}.
  \item \textbf{Averaging benefit on long plateaus.} If $k$ is uniform over a long plateau and $\OPT$ does not change, then running $k$-means++ with parameter $k$ is akin to oversampling relative to the \emph{effective} cluster count on that scale. Using Theorem~5.1 from the reference with $\varepsilon \approx |I_j|/k$ suggests an $O(\log\log k_0)$ \emph{conditional} bound on that plateau.
  \item \textbf{Bounding the mass of short plateaus.} Either there are few short plateaus at each scale (so their contribution in the uniform average is small), or there are many in a row; in the latter case, we can try to invoke a strengthened oversampling statement (see next) to regain a $\log\log$ bound.
\end{itemize}

\paragraph{2.\ A coverage lemma for the ``heavy'' part}
\label{subsec:coverage}
Aim to prove a statement of the form:

\medskip
\noindent\emph{Hypothesis (strengthening of Thm.~5.1).}
Let $k=k_1{+}k_2$. Suppose $\OPT(k_1) \ge k^{10}\,\OPT(k)$ (a strong scale separation).
Run $k$-means++ for $k'=k_1 + (1+\epsilon)k_2 $ steps. Then, with probability $1 - 1/k^{O(1)}$, the first $k_1$ steps place one center in each of the $k_1$ ``heavy'' clusters before any collision, and overall the returned cost is $O(\log(1/\varepsilon))\cdot \OPT(k)$.

\medskip
\noindent\textbf{Why plausible.}
Under strong separation, the contribution to $D^2$ from uncovered heavy clusters dominates. By the standard $k$-means++ potential argument, the chance to hit an already covered heavy cluster in the early steps is small. 