
The task is to show $\Omega(\log^2 n)$ lower bound for sorting bit-reversal sequence in a model defined next. (Alternatively, provide $o(\log^2 n)$ algorithm if there is one)


\section{Model explanation and the statement of the problem}

\paragraph{Model (one step).}
We start with an array $A$ containing the integers $0,1,\dots,n-1$ in some order.
A single \emph{step} of the algorithm consists of:
\begin{enumerate}
  \item Partition $A$ into disjoint, consecutive intervals that cover $A$.
  \item Select some disjoint pairs of \emph{adjacent} intervals.
  \item For each selected pair, \emph{swap} the two intervals as whole blocks (these swaps occur in parallel).
  \item Unmatched intervals remain in place.
\end{enumerate}
The partition and the chosen pairs may depend on the current contents of $A$.

\paragraph{Example of one step ($n=16$).}
Start with
\[
[0,1,2,3,\;\;4,5,6,7,\;\;8,9,10,11,12,\;\;13,14,\;\;15].
\]
Partition into $I_1=[0,1,2,3]$, $I_2=[4,5,6,7]$, $I_3=[8,9,10,11,12]$, $I_4=[13,14]$, $I_5=[15]$.
Match $(I_1,I_2)$ and $(I_4,I_5)$; leave $I_3$ unmatched. After the step:
\[
[4,5,6,7,\;\;0,1,2,3,\;\;8,9,10,11,12,\;\;15,\;\;13,14].
\]

\paragraph{Bit-reversal order}
Let $n=2^d$. The \emph{bit-reversal order} lists $0,1,\dots,n-1$ and reverses the $d$-bit representations to obtain the input permutation.

\paragraph{Example ($n=16$, $d=4$).}
\[
\underbrace{0,\;8,\;4,\;12,\;2,\;10,\;6,\;14,\;1,\;9,\;5,\;13,\;3,\;11,\;7,\;15}_{\text{bit-reversal sequence}}.
\]
Write bits (from MSB $b_3$ to LSB $b_0$) under the sequence:
\[
\begin{array}{l|cccccccccccccccc}
\text{values} & 0&8&4&12&2&10&6&14&1&9&5&13&3&11&7&15\\\hline
b_3\;(\text{MSB}) & 0&1&0&1&0&1&0&1&0&1&0&1&0&1&0&1\\
b_2 & 0&0&1&1&0&0&1&1&0&0&1&1&0&0&1&1\\
b_1 & 0&0&0&0&1&1&1&1&0&0&0&0&1&1&1&1\\
b_0\;(\text{LSB}) & 0&0&0&0&0&0&0&0&1&1&1&1&1&1&1&1\\
\end{array}
\]
For $n=8$ ($d=3$) the bit-reversal sequence is $[0,4,2,6,1,5,3,7]$ with analogous bit rows:
$b_2:01010101$, $b_1:00110011$, $b_0:00001111$ (truncated to $8$ entries).

\paragraph{The problem}
Your task is to prove that any algorithm in our model that sorts this bit-reversal sequence requires $\Omega(\log^2 n)$ steps. Alternatively, if there is an algorithm that requires only $o(\log^2 n)$ steps, prove that. 

\section{Baseline results}

\paragraph{Counting lower bound: $\Omega(\log n)$ steps.}
There are $n!$ possible inputs, requiring $\log_2(n!)=n\log_2 n - \Theta(n)$ bits to distinguish.
One step is determined by a partition plus a choice of which adjacent interval-pairs to swap, all describable with $O(n)$ bits, i.e.\ at most $2^{O(n)}$ possibilities per step.
Hence each step injects only $O(n)$ bits of ``entropy'', and sorting from scratch needs $\Theta(n\log n)$ bits, implying $\Omega(\log n)$ steps lower bound for any general sorting algorithm. 

\paragraph{Upper bounds: $O(\log^2 n)$ steps.}
Two standard paradigms adapt to this model.

\medskip\noindent
\textbf{(A) Merge-sort style.}
Sort blocks of size $2$, then $4$, then $8$, etc., doubling each stage. The key primitive is \emph{merging} two sorted sequences.
Merging can be done in $O(\log n)$ steps by recursive median partitioning:
pick the median $m$ of the union, partition both sequences into $<m$ and $\ge m$, use $O(1)$ flips to place all $<m$ before all $\ge m$, and recurse on the two halves in parallel.
Depth $O(\log n)$; each merge takes $O(\log n)$ steps $\Rightarrow O(\log^2 n)$ total.

\medskip\noindent
\textbf{(B) Quicksort style.}
Given pivot $p$, color items red if $<p$ and blue otherwise. View the array as maximal monochromatic blocks.
One step can reduce the number of red/blue blocks by a constant factor (by flipping adjacent red/blue block pairs to coalesce like colors).
Thus partitioning by $p$ takes $O(\log n)$ steps; recurse on the two sides $\Rightarrow O(\log^2 n)$ overall.

\subsection{Two specialized $O(\log^2 n)$ strategies.}

\medskip\noindent
\textbf{MSB-first (quicksort-like).}
The MSB row is $0101\ldots$. Using the red/blue block-reduction, in $O(\log n)$ steps group all MSB$=0$ items left and MSB$=1$ right.
For $n=16$ this yields:
\[
\text{left: }[0,4,2,6,1,5,3,7]\qquad
\text{right: }[8,12,10,14,9,13,11,15].
\]
Each side is the bit-reversal sequence on $n/2$ elements (one fewer bit). Recurse for depth $\log n$; each level costs $O(\log n)$ steps $\Rightarrow O(\log^2 n)$.

\medskip\noindent
\textbf{LSB-first (interlacing).}
Initially LSB has eight $0$'s then eight $1$'s, so the sequence splits into quarters
\[
[0,8,4,12]\;|\;[2,10,6,14]\;|\;[1,9,5,13]\;|\;[3,11,7,15].
\]
Flip the second and third quarters to interlace:
\[
[0,8,4,12]\;|\;[1,9,5,13]\;|\;[2,10,6,14]\;|\;[3,11,7,15].
\]
Continue interlacing for $O(\log n)$ rounds to obtain
\[
[0,1,\;8,9,\;4,5,\;12,13,\;2,3,\;10,11,\;6,7,\;14,15],
\]
which forms sorted pairs $(0,1),(8,9),\dots$. Treat pairs as super-items and repeat to form sorted $4$-tuples, then $8$-tuples, etc.
$\log n$ macro-levels, each $O(\log n)$ steps $\Rightarrow O(\log^2 n)$.

\subsection{Structural $\Omega(\log n)$ lower bounds beyond counting}
The counting lower bound ($\Omega(\log n)$) is too weak for $\Omega(\log^2 n)$. Two task-specific $\Omega(\log n)$ lower bounds are relevant.

\paragraph{Merging requires $\Omega(\log n)$ steps.}
Consider
\[
[0,2,4,6,8,10,12,14 \;\mid\; 1,3,5,7,9,11,13,15].
\]
Color evens red and odds blue. Start: two monochromatic blocks (R$|$B).
Target: fully interlaced (R,B,R,B,$\dots$), i.e.\ $\Theta(n)$ blocks.
\emph{Observation:} In one step, only boundaries where a red block meets a blue block can create more alternations; each matched adjacent-pair affects only $O(1)$ such boundaries.
Thus the number of monochromatic blocks can grow by at most a constant factor per step.
Going from $2$ to $\Theta(n)$ blocks needs $\Omega(\log n)$ steps.

\paragraph{Pivoting (coarsening) requires $\Omega(\log n)$ steps.}
Color items $<n/2$ red and $\ge n/2$ blue. For $n=16$ in bit-reversal order the pattern is R,B,R,B,\dots (many small blocks).
Goal: all reds then all blues (two blocks).
By the same boundary-locality, a step can reduce the number of blocks by only a constant factor, so $\Omega(\log n)$ steps are necessary.

\paragraph{Significance.}
If merging or pivoting were doable in $O(1)$ steps, the corresponding $O(\log^2 n)$ algorithm would become $O(\log n)$.
Hence any $\Omega(\log^2 n)$ lower bound must \emph{internally} show that both primitives inherently cost $\Omega(\log n)$ in this model.

\section{Ideas for how a proof might look like}

\paragraph{Important note}
You don't have to follow these ideas, take them only as a very speculative guess for how a solution might look like. 

Write $n=2^d$.
For an arrangement $A$ and bit level $j\in\{0,\dots,d-1\}$ (LSB $j=0$), let $B_j(A)$ be the number of monochromatic runs when items are colored by their $j$-th bit.

\paragraph{Values for bit-reversal vs.\ sorted.}
For the bit-reversal input $A_0$:
\[
B_j(A_0)=2^{j+1}\quad (j=0,1,\dots,d-1).
\]
For the sorted order $A_\star$:
\[
B_j(A_\star)=\frac{n}{2^j}=2^{d-j}\quad (j=0,1,\dots,d-1).
\]
Define the per-level log-gap
\[
D_j(A):=\big|\log_2 B_j(A)-\log_2 B_j(A_\star)\big|.
\]
Then
\[
D_j(A_0)=\big|(j+1)-(d-j)\big|=\big|2j+1-d\big|,\qquad
\sum_{j=0}^{d-1} D_j(A_0)=\Theta(d^2)=\Theta(\log^2 n).
\]

\paragraph{Goal.}
Prove a per-step progress cap showing that the potential
\[
\Phi(A)\;:=\;\sum_{j=0}^{d-1} D_j(A)
\]
drops by at most $O(1)$ per step. Since $\Phi(A_0)=\Theta(\log^2 n)$ and $\Phi(A_\star)=0$, this would yield an $\Omega(\log^2 n)$ lower bound.

\subsection*{Two lemmas that could be helpful if true (informal)}

\paragraph{Lemma 1 (per-level constant-factor cap, informal).}
For any fixed level $j$, one step can change $B_j$ by at most a constant multiplicative factor.
\emph{Rationale:} Only boundaries between opposite colors (at level $j$) can modify $B_j$, and each matched adjacent-interval flip perturbs $B_j$ by $O(1)$.

\paragraph{Lemma 2 (cross-level scarcity, informal).}
A matched boundary (between two swapped intervals) is only ``useful'' for $O(1)$ nearby bit levels, determined by the scale of the intervals. Consequently, in one step, only $O(1)$ levels can experience a constant-factor change in $B_j$.
\emph{Rationale:} Progress at level $j$ needs boundaries aligned at the dyadic scale $2^j$; a boundary created at a given geometric scale cannot simultaneously serve many levels.

\paragraph{Consequence.}
Combining Lemmas~1 and~2, $\Phi$ decreases by at most $O(1)$ per step, forcing $\Omega(\log^2 n)$ steps to sort the bit-reversal input.

\paragraph{Important note}
We retiterate that this section is just a speculative attempt not know to lead to a solution. 